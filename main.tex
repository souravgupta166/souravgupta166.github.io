\documentclass{article}
\usepackage{hyperref}
\usepackage{geometry}
\geometry{a4paper, margin=1in}

\title{Computer Science Portfolio Report}
\author{Sourav Gupta}
\date{05th July 2024}

\begin{document}
\maketitle

\section{Introduction}

This paper describes how my GitHub Pages computer science portfolio was created and put into practice. The portfolio, which includes a LaTeX-created CV and several projects showcasing my expertise in software development and other pertinent fields, is intended to demonstrate my technical talents. The website emphasizes a clear layout, user-friendliness, and efficient documentation, providing a polished platform for me to showcase my work to prospective employers. It has sections that are necessary, like a projects page, CV page, and homepage. This research examines the portfolio's strengths and potential areas for future development while providing specifics on its structure, design choices, and technologies employed.


\section{Portfolio Structure}
The portfolio website is structured as follows:
\begin{itemize}
    \item \textbf{Home Page} (\href{https://souravgupta166.github.io/}{Link}): Contains a brief introduction.
    \item \textbf{CV Page} (\href{https://souravgupta166.github.io/Sourav_Latex_CV.pdf}{Link}): Displays and provides a downloadable link to my CV.
    \item \textbf{Projects Page} (\href{https://souravgupta166.github.io/PowerBI%20Dashboards.pdf}{Link}): Lists my projects with descriptions and links.
    
\end{itemize}

\section{
Design Decisions for Portfolio Website}
1. Use of Bootstrap Framework:
\\
Justification: Bootstrap significantly simplifies the creation of a responsive design, ensuring that the site adapts seamlessly to various screen sizes, from desktops to mobile devices. This enhances the accessibility and usability of the portfolio.

\\

2- Minimalistic and Clean Layout:

Justification: A minimalistic approach reduces visual clutter, making it easier for hiring managers to focus on the showcased projects and CV. The clear navigation structure allows quick access to different sections, improving the overall user experience.

\\

3- Centralized Navigation Bar

A centralized navigation bar enhances user navigation by providing intuitive and immediate access to key sections. The use of dropdowns for contact information keeps the main menu streamlined while making additional details readily available.

\\

4. Background and Styling Choices

Justification: The color scheme complements the professional tone of the portfolio, while the background image adds aesthetic value, making the About Me section more engaging. The CSS styling ensures a consistent and visually appealing presentation.

\\

5. Document Linking

Justification: Direct linking facilitates easy access to my CV and projects, allowing hiring managers to quickly review my qualifications and work samples without navigating away from the site.



\section{Tools and Technologies}
\begin{itemize}
    \item \textbf{HTML/CSS}: HTML5 is the latest standard for HTML and offers advanced features for structuring web content, including semantic elements that improve accessibility and SEO. Its wide adoption ensures compatibility across all modern web browsers and devices, which is crucial for reaching a broad audience.
    
    \\
    
    \item \textbf{GitHub Pages}:  GitHub Pages provides a free and user-friendly platform for hosting static websites directly from a GitHub repository. It integrates seamlessly with Git for version control and deployment, allowing efficient updates and maintenance of the portfolio. This choice also leverages GitHub’s widespread use in the industry, making it a familiar and reliable platform for potential employers.
    \\
    
    \item \textbf{LaTeX}: LaTeX is a high-quality typesetting system that is widely used for producing scientific and technical documents. It allows precise control over document layout and formatting, which is ideal for creating professional CVs and reports. LaTeX also supports various document structures and citation formats, making it suitable for comprehensive documentation.

    \item \textbf{CSS}: CSS3 provides powerful styling capabilities that enable the creation of a visually appealing and consistent design. Features such as Flexbox and Grid Layout facilitate responsive design, ensuring the portfolio looks good on various screen sizes. CSS3 also supports animations and transitions that enhance user experience.
    
\end{itemize}

\section{Strengths and Weaknesses}
\begin{itemize}
    \item \textbf{Strengths}: Easy to navigate, professional appearance, well-documented.
    \item \textbf{Weaknesses}: Could benefit from additional interactive features and more detailed project descriptions.
\end{itemize}

\section{Future Improvements}
I plan to add more interactivity, such as a contact form and more detailed descriptions of projects, including demo links or embedded videos.

\end{document}
